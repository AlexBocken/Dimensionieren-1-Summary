\section{Kriechen}
    \subsection{Ursachen}
        Kriechen tritt bei metallischen Stoffen bei $ T < T_{schmelz} \cdot 0.3  $ auf.
        Bewegung von Versetzungen, Gleit- und Diffussionsphänomene. 
        
        $\rightarrow$ inelastische Deformation  
        
    \subsection{Relaxation}
        \subsubsection{Nach Norton}
            ges: $\sigma(t)$ geg: $\varepsilon_{cr}, \varepsilon{el}$
        
            $\varepsilon_{cr} + \varepsilon_{el} = \varepsilon = $ const.
            
            $\dot{\varepsilon_{cr}} + \dot{\varepsilon_{el}} = 0 \rightarrow A\sigma^n + \frac{1}{E}\dot{\sigma} = 0$ 
            
            Separation der Variablen und Integrationskonstante mit Anfangsspannung bestimmen.
    \subsection{Kriechschädigung}
            \TODO{}
    
    \subsection{Methoden zur Dimensionierung}
        
\TODO{}