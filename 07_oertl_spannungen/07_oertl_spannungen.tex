\section{Ermüdung mit örtlichen Spannungen(FKM Kap. 3)}
    \subsection{Lastkollektive Bilden}
    Aus elast. FEM Hauptspannungen an den kritischen Stellen bestimmen.
    aus $\sigma_1(t)-t$ Diagramm Belastungskollektiven einteilen. $\rightarrow \sigma_{1a,1}, \sigma_{1m,1}$; $\sigma_{1a,i}, \sigma_{1m,i}$ bestimmen.
    \\1. Subskript: Welche HS-Richtung (index tiefer $\rightarrow$ HS höher); 
    \\2. Subskript: Welches Kollektiv (index tiefer $\rightarrow$ $\sigma_a$ höher)
    \subsection{Ausnutzungsfaktor für jede HS}
        Ziel: $a_{\textrm{BK},i} < 1$ mit $\displaystyle a_{\textrm{BK},i} = \frac{\sigma_{i,a,1}}{\sigma_{\textrm{BK}}/J_D}$; \quad $J_D=J_S\cdot J_F \cdot J_G$ 
        \\Durchführen für alle Hauptspannungen.
    \subsection{Berechnung von $\sigma_{\textrm{BK}}$}
        \[\sigma_{\textrm{BK}} = K_{\textrm{BK}} \cdot K_{\textrm{AK}} \cdot \frac{1}{K_{\textrm{WK}}} \cdot \sigma_w \]
        \subsubsection{Kerben \& Oberflächen}
            \[K_{\textrm{WK}}=\left(\left[1+\frac{1}{\widetilde{K}_F}\left(\frac{1}{K_R}-1\right)\right]\frac{1}{K_v}\right)\frac{1}{\eta_{\sigma}}\]
            $\widetilde{K}_F$: Rauheitsfaktor;   $K_R$: Sensitivität auf Rauheit bei Kerben;     $K_v$: Randschichtfaktor;   $\eta_{\sigma}$: Stützzahl aus Kerbwirkung
        \subsubsection{Einfluss der Mittelspannung $\sigma_m$}
            \[K_{\textrm{AK},i}=\frac{1}{1+M\frac{\sigma_{m,i}}{\sigma_{a,i}}} \qquad\textrm{(aus Haigh-Diagramm)}\]
        \subsubsection{Beschränkte Schwingungszahl}
            Einstufenkollektiv: $\displaystyle K_{\textrm{BK}}= \frac{\sigma_{a}(N=N_L)}{\sigma_w}$\\
            Allgemein:\qquad\qquad $\displaystyle K_{\textrm{BK}}=\left(\frac{A\cdot D_{\textrm{m}}\cdot N_{\textrm{D}}}{\bar{N}}\right)^{1/K}$
            \[A = \frac{1}{\frac{n_1}{\bar{N}}\cdot\left(\frac{\sigma_{a,R,1}}{\sigma_{a,1}}\right)^{K}+\frac{n_2}{\bar{N}}\cdot\left(\frac{\sigma_{a,R,2}}{\sigma_{a,1}}\right)^{K}};\quad \sigma_{a,R,i}=\sigma_{a,i}\cdot\frac{K_{\textrm{AK},1}}{K_{\textrm{AK},i}}\]
            $\bar{N}$: $\Sigma$ Zyklen; \qquad $D_{\textrm{m}}$: Schadensgrenze
    \subsection{BSP}
        \TODO{}